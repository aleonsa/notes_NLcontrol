\chapter{Forma de Controlador}
\section{Forma de Controlador}
\defn{Forma de Controlador}{
	Se dice que un sistema no lineal (con múltiples entradas) está en la \gls{fc} si se puede escribir como
	\begin{equation}
		\dot{x} = Ax + B\gamma(x)[u - \alpha(x)] \, ,
		\label{eq:forma_controlador}
	\end{equation}
	donde
	\begin{itemize}
		\item $(A, B)$ es controlable, $A \in \mathbb{R}^{n \times n}$, $B \in \mathbb{R}^{n \times p}$.
		\item $\gamma(x)$, con $\gamma: D \subset \mathbb{R}^n \rightarrow \mathbb{R}^{p\times p}$, es una matriz no singular para cada $x \in D$.
	\end{itemize}
}

\rmkb{Note que todas las no linealidades están en el mismo canal que la entrada. La dinámica no lineal del sistema se puede manipular libremente.}

\fact{
	Un sistema que está en la \gls{fc} se puede linealizar exactamente mediante retroalimentación de los estados:\\

	Si se elige
	\begin{equation*}
		u = \alpha(x) + \gamma^{-1}(x) v \, ,
	\end{equation*}
	donde
	\begin{itemize}
		\item $\gamma^{-1}(x)$ es la inversa de $\gamma(x)$, y
		\item $v$ es la entrada de control,
	\end{itemize}
	entonces la dinámica del sistema en lazo cerrado es

	\begin{equation*}
		\dot{x} = Ax + Bv \, ,
	\end{equation*}
	que es lineal en los estados y en la entrada $v$.
}
\defn{Linealizable por Retroalimentación}{
	Se dice que un sistema no lineal
	\begin{equation*}
		\dot{x} = f(x) + G(x)u \, ,
	\end{equation*}
	donde $f: D \rightarrow \mathbb{R}^n$ y $G: D \rightarrow \mathbb{R}^{n \times p}$ son suficientemente suaves en un dominio $D \subseteq \mathbb{R}^n$, es \textbf{Linealizable por Retroalimentación} (o Linealizable Entrada-Estados) si existe un difeomorfismo $T : D \rightarrow \mathbb{R}^n$ tal que
	\begin{itemize}
		\item $D_z = T(D)$ contiene el origen, y
		\item el cambio de variable $z=T(x)$ lleva al sistema a la \gls{fc}.
	\end{itemize}
}

\propp{
	El sistema de dimensión $n$ y con una entrada ($p=1$)
	\begin{equation*}
		\dot{x} = f(x) + g(x)u
	\end{equation*}
	puede ser transformado a la \gls{fc} mediante una transformación de estados $z=T(x)$ si y solo si existe una función (de salida) $h(x)$ tal que el sistema \gls{siso}
	\begin{equation*}
		\begin{aligned}
			\dot{x} & = f(x) + g(x)u \\
			y       & = h(x)
		\end{aligned}
	\end{equation*}
	tenga grado relativo completo, i.e., $\rho = n$.
}{
	$\Leftarrow$: Si el sistema tiene grado relativo $\rho = n$, entonces se puede llevar a la \gls{fn}, que en este caso se reduce a la \gls{fc}, pues no hay porción $\eta$
	\begin{equation*}
		\begin{aligned}
			\dot{\xi} & = A_c \xi + B_c \gamma(x)[u - \alpha(\xi)] \\
			y         & = C_c \xi
		\end{aligned}
	\end{equation*}
	\\
	$\Rightarrow$: Si el sistema se puede transformar a la \gls{fc} mediante una transformación $z=S(x)$, entonces
	\begin{equation*}
		\dot{z} = Az + B \bar{\gamma}(z)[u - \bar{\alpha}(z)].
	\end{equation*}
	Como $(A,B)$ es controlable, entonces existe una matriz regular $M \in \mathbb{R}^{n \times n}$ tal que
	\begin{equation*}
		\begin{aligned}
			MAM^{-1} & = A_c + B_c \lambda^T \\
			MB       & = B_c
		\end{aligned} \, ,
	\end{equation*}
	donde
	\begin{equation*}
		A_c = \begin{bmatrix}
			0      & 1      & 0      & \ldots & 0      \\
			0      & 0      & 1      & \ldots & 0      \\
			\vdots & \vdots & \vdots & \ddots & \vdots \\
			0      & 0      & 0      & \ldots & 1      \\
			0      & 0      & 0      & \ldots & 0
		\end{bmatrix} \; , \quad B_c = \begin{bmatrix}
			0      \\
			0      \\
			\vdots \\
			0      \\
			1
		\end{bmatrix} \; .
	\end{equation*}
	Por lo tanto, la transformación
	\begin{equation*}
		\xi = Mz = MS(x) \overset{\Delta}{=} T(x)
	\end{equation*}
	transforma al sistema a la forma
	\begin{equation*}
		\dot{\xi} = A_c \xi + B_c \gamma(x)[u - \alpha(x)] \, ,
	\end{equation*}
	donde
	\begin{equation*}
		\gamma(x) = \bar{\gamma}(x) \, , \quad \alpha(x) = \bar{\alpha}(x) - \lambda^T M \dfrac{S(x)}{\gamma(x)} \, .
	\end{equation*}
	Eligiendo
	\begin{equation*}
		y = C_c \xi = \xi_1 = T_1(x) = h(x)
	\end{equation*}
	el sistema tiene \gls{gr} $\rho = n$.
}

\section{Condiciones de existencia de $h(x)$}
Ahora nos concentraremos en el problema de hallar una salida con \gls{gr} $\rho = n$.\\

Trataremos de encontrar una caracterización a partir de la cual, con solo mirar la $f(x)$ y $g(x)$, y haciendo ciertas operaciones, decidir si existe un difeomorfismo que lleve a la \gls{fc}.\\

¿Cómo decidir si existe una función suave $h(x)$ tal que
\begin{equation}
	\begin{aligned}
		L_g L_f^{i-1} h(x) & = 0, \quad i = 1, 2, \ldots, n-1; \quad \forall x \in D_0 \\
		L_g L_f^{n-1} h(x) & \neq 0
	\end{aligned}
	\label{eq:edp_h}
\end{equation}
y
\begin{equation*}
	T(x) = \begin{bmatrix}
		h(x)     \\
		L_f h(x) \\
		\vdots   \\
		L_f^{n-1} h(x)
	\end{bmatrix}
\end{equation*}
sea un difeomorfismo?\\

Note que esto termina siendo un problema de resolver una \gls{edp}, que en general es muy complicado. Podemos conformarnos inicialmente en saber si \textbf{existe} una función $h(x)$ que cumpla con las condiciones.\\

Para este fin, es necesario introducir herramientas adicionales.\\

\subsection{El paréntesis de Lie}
\defn{Paréntesis de Lie}{
	El \textbf{paréntesis de Lie} de dos campos vectoriales $f$ y $g$ es un tercer campo vectorial definido como
	\begin{equation}
		[f, g] = \dfrac{\partial g(x)}{\partial x} f(x) - \dfrac{\partial f(x)}{\partial x} g(x) \, .
		\label{eq:lie_bracket}
	\end{equation}
}
Para simplificar la notación, se puede escribir
\begin{equation}
	\begin{aligned}
		 & ad_f^0 g(x) = g(x)                      \\
		 & ad_f g(x) = [f, g](x)                   \\
		 & ad_f^k g(x) = [f, ad_f^{k-1} g](x) \, .
	\end{aligned}
	\label{eq:lie_bracket_simplified}
\end{equation}
Además, algunas propiedades del paréntesis de Lie son
\begin{itemize}
	\item $ [f, g] = -[g, f] $
	\item  Para campos vectoriales constantes $ f, g, \; [f, g] = 0$
\end{itemize}

\subsection{Distribuciones}
Partiendo de la idea de un campo escalar, el cual es una función que asigna un número a cada punto del espacio, podemos decir, de manera informal, que una distribución es un campo en el que a cada punto del espacio se asigna un subespacio. Usualmente, lo hacemos asignando a cada punto un conjunto de vectores que generan el subespacio.

\defn{Distribución}{
	Para los campos vectoriales $f_1, f_2, \ldots, f_k$ en $D \subset \mathbb{R}^n$, sea
	\begin{equation*}
		\Delta(x) = \text{span}\{f_1(x), f_2(x), \ldots, f_k(x)\}
	\end{equation*}
	el espacio vectorial generado por los vectores $f_1(x), f_2(x), \ldots, f_k(x)$.
	La colección de espacios vectoriales $\Delta(x)$ para todo $x \in D$ se llama una \textbf{Distribución} y nos referimos a ella como
	\begin{equation*}
		\Delta = \text{span}\{f_1, f_2, \ldots, f_k\} \, .
	\end{equation*}
	\begin{itemize}
		\item Si la $\text{dim}(\Delta(x)) = k$ para todo $x \in D$, se dice que $\Delta$ es una \textbf{Distribución no singular} (o regular) en $D$, generada por los campos vectoriales $f_1, f_2, \ldots, f_k$.
		\item La distribución es \textbf{Involutiva} si
		      \begin{equation*}
			      g_1 \in \Delta \; \text{y} \; g_2 \in \Delta \quad \Rightarrow \quad [g_1, g_2] \in \Delta
		      \end{equation*}
	\end{itemize}
}

\lem{}{
	Si $\Delta$ es una distribución regular en $D$, generada por $f_1, f_2, \ldots, f_k$, entonces es involutiva si y solo si
	\begin{equation}
		[f_i, f_j] \in \Delta, \quad \forall 1 \leq i, j \leq k \, .
		\label{eq:involutiva}
	\end{equation}
}

\thm{Teorema de Linealización}{
	\label{thm:linealizacion}
	El sistema de dimensión $n$ con una sola entrada \gls{si}
	\begin{equation}
		\dot{x} = f(x) + g(x)u
		\label{eq:si}
	\end{equation}
	es transformable a la \gls{fc} si y solo si existe un dominio $D_0$ tal que
	\begin{enumerate}
		\item $\text{rank} [g(x), ad_f g(x), \ldots, ad_f^{n-1} g(x)] = n, \quad \forall x \in D_0$, y
		\item $\text{span} \{g(x), ad_f g(x), \ldots, ad_f^{n-2} g(x)\}$ es involutiva en $D_0$.
	\end{enumerate}
}

\ex{
\[
	\dot{x} = \begin{bmatrix}
		a \sin x_2 \\
		-x_1 ^2
	\end{bmatrix} + \begin{bmatrix}
		0 \\
		1
	\end{bmatrix} u
\]
\[
	\text{ad}_f g(x) = [f,g] = - \dfrac{\partial f}{\partial x}g = \begin{bmatrix}
		-a \cos x_1 \\
		0
	\end{bmatrix}
\]
Note que el procedimiento para encontrar difeomorfismo de estados que transforme al istema a su \gls{fc} consiste en los siguientes pasos:
\begin{enumerate}
	\item A partir de $f$ y $g$, construir los campos vectoriales
	      \[
		      \text{ad}_f g(x), \; \ldots \; , \text{ad}_f^{n-2} g(x), \; \text{ad}_f^{n-1} g(x)
	      \]
	      y verificar las condiciones (i) y (ii) del Teorema de Linealización (\ref{thm:linealizacion}).
	\item Si ambas condiciones son satisfechas, halle $h(x)$ resolviendo el sistema de \gls{edp} (\ref{eq:edp_h}).
	\item Defina
	\[ \alpha(x) = -\dfrac{L_f^n h(x)}{L_g L_f^{n-1} h(x)} \; , \quad \gamma(x) = L_g L_f h(x) \]
	\item Halle la transformación
	\[ T(x) = \begin{bmatrix}
		h(x)     \\
		L_f h(x) \\
		\vdots   \\
		L_f^{n-1} h(x)
	\end{bmatrix} \]
\end{enumerate}
}
\section{Tres consecuencias del Teorema de Linealización}
Tres consecuencias importantes del teorema anterior serán presentadas \cite{marino1995}\cite{isidori1995}.
\cor{
    Un sistema no lineal en el plano ($n=2$) con una sola entrada
    \begin{equation*}
        \dot{x} = f(x) + g(x)u
    \end{equation*}
    es Linealizable por Retroalimentación (es localmente transformable a la \gls{fc}) en una vecindad del origen si y solo si su aproximación lineal (linealización) en el origen
    \begin{equation*}
        \dot{x} = \dfrac{\partial f(0)}{\partial x}x + g(0)u \overset{\Delta}{=} Ax + Bu
    \end{equation*}
    es controlable.
}

\cor{
    La controlabilidad de la linealización para sistemas de orden $n \geq 2$ es una condición necesaria, más no suficiente de Linealizabilidad por Retroalimentación.

    Si el sistema no lineal con una sola entrada
    \begin{equation*}
        \dot{x} = f(x) + g(x)u
    \end{equation*}
    es Linealizable localmente por Retroalimentación (es localmente transformable a la \gls{fc}) en una vecindad del origen, entonces su linealización en el origen
    \begin{equation*}
        \dot{x} = \dfrac{\partial f(0)}{\partial x}x + g(0)u \overset{\Delta}{=} Ax + Bu
    \end{equation*}
    es controlable, es decir,
    \begin{equation*}
        \text{rank}[b, Ab, \ldots, A^{n-1}b] = n \, .
    \end{equation*}
}

\cor{
    El sistema en Forma Triangular 
    \begin{equation*}
        \begin{aligned}
        \dot{x}_i &= \phi_i(x_1, \ldots, x_i) + x_{i+1}u, \quad i = 1, 2, \ldots, n-1 \\
        \dot{x}_n &= \phi_n(x_1, \ldots, x_n) + u,
        \end{aligned}
    \end{equation*}
    en el cual, $\phi_1, \ldots, \phi_n$ son funciones suficientemente suaves, tales que $\phi_i(0) = 0, 1\leq i\leq n$ es Linealizable por Retroalimentación.
}

